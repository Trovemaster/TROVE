
\chapter{Theory}
\label{chap:theory}

In this chapter the theory underlying the TROVE program will be described. Since the original TROVE paper many new 
features have been added to the code and described in publications as used. Here all of the theoretical framework 
for the main aspects of TROVE shall be described in one place. 

The aim of this chapter is not to derive and explain every technical aspect of TROVE per say but to give users knowledge of the
underlying concepts of TROVE. This is especially useful for those using it as a `black box' (and for PhD Vivas!). For full
technical details of derivations and theoretical concepts, the reader should consult the references as given and the 
TROVE source code.

\section{The TROVE Approach}
The Schr\"{o}dinger equation for a molecule within the Born-Oppenheimer (BO) approximation can be written in laboratory-fixed 
$XYZ$ Cartesian coordinates as 
\begin{equation}
\label{eq.schrodiger_lab_cart}
\left(-\frac{\ \hbar^2}{2} \sum_{i=1}^N \frac{1}{m_i} \nabla^2_i + V \right) \Psi_{trv} = E_{trv} \Psi_{trv}
\end{equation}
where nucleus $i$ has mass $m_i$ and coordinates $(R_{iX},R_{iY},R_{iZ})$. $\nabla^2_i = \partial ^2 / \partial R_{iX}^2 +
\partial ^2 / \partial R_{iY}^2  + \partial ^2 / \partial R_{iZ}^2$ is the kinetic energy for each nucleus and $V$ is the 
potential energy, which is equal to the electronic energy within the the BO approximation (and absence of electric and 
magnetic fields). $\Psi_{trv}$ and $E_{trv}$ are the wavefunction and energy respectively. 

Although this equation has a simple mathematical form, it is not suited for actual solution. The label $trv$ on the energies
and wavefunction stands for translation, rotation and vibration. In this coordinate system these motions are all coupled. 
Translational motion can be exactly separated from rotation and vibration (indeed, for spectroscopy we are usually not 
interested in the overall translation of a molecule through space). Rotational and vibrational motion are often fairy 
weakly coupled and coordinates can be chosen to exploit that fact. Maximal separation of all of these motion makes the 
solution of the nuclear Schr\"{o}dinger equation simpler.

To carry out this decoupling, a more suitable coordinate system is required. This usually involves introducing a molecule-fixed
(or body-fixed) axis system, $xyz$ with the origin at the molecule's centre of mass. Thus the nuclear locations are some vector
from the centre of mass rather than from some origin in space in the $XYZ$ system. If the molecule is rotating then the molecule-fixed axis will rotate with the molecule. 

Defining a molecule-fixed axis system immediately introduces a problem however due to the variety of shapes and sizes of 
molecules. For each molecule type (e.g. trigonal pyramidal, tetrahedral, etc) there are different choices of coordinate system
which lead to different kinetic energy operators (and different computer programs for their solution). Normal coordinates 
are an exception and can be defined in a general manner but are best suited for small amplitude vibrations near a 
molecule's equilibrium structure. 

TROVE takes a different approach by numerically constructing the kinetic energy operator for a given molecule and axis system.
This is achieved using a Taylor expansion of the Hamiltonian in terms of internal coordinates of the molecule.
This allows TROVE to be used for a wide variety of molecules as seen in chapter \ref{chap:molecules}. The actual construction
of the Hamiltonian can be used in a rather `black box' manner, with the user only needing to define coordinate
transforms and so on (see chapter \ref{chap:newmol}). 

A particular strength of TROVE is the ability to calculate eigenfunctions and eigenvalues of high angular momentum quantum
number $J$ by minimising the coupling of the vibrational and rotational motion. Access to high $J$s is crucial for the 
simulation of molecular spectra, especially at high temperatures where lots of states are populated. 

\section{Numerical Construction of Kinetic Energy Operator}
\label{sec.numerical_T}

To construct the kinetic energy operator TROVE expresses the Hamiltonian in equation \ref{eq.schrodiger_lab_cart} 
in terms of the generalised coordinates
\begin{equation}
\label{eq.gen_coord}
\Xi = \left(R_X^{CM},R_Y^{CM},R_Z^{CM},\theta,\phi,\chi,\xi_1,\xi_2 \cdots ,\xi_{3N-6} \right)
\end{equation}
where $R_F^{CM} = \sum_{i=1}^N m_iR_{iF} / \sum_{j=1}^N m_j$ $(F=X,Y,Z)$ is the $F$-coordinate of the nuclear centre of mass;
these three coordinates describe translation of the molecule through space. The three Euler angles ($\theta,\phi,\chi$) 
define the orientation of the $xyz$ molecule-fixed axis relative to lab-fixed $XYZ$ and thus define overall 
rotation (for a description of Euler angles 
see Bunker and Jensen \cite{98BuJexx} or Zare's book on angular momentum \cite{88Zare.book} ). 
The $\xi_n$ coordinates described vibration. There are $3N - 6$ of these for a non-linear molecule and they can be defined
in whatever manner is convenient (see below).

The transformed kinetic energy operator $\hat{T}$ is essentially a quadratic form in the generalised momenta
 ( recall that $-\frac{\hbar^2}{2m} \frac{\partial^2 }{ \partial x^2 } = \frac{1}{2m} \left( -i \hbar 
\frac{\partial}{\partial x} \right)^2 = \frac{\hat{p}^2}{2m}$ )
\begin{equation}
\label{eq.gen_momenta}
\hat{\Pi} = \left(\hat{P}_X^{CM}, \hat{P}_Y^{CM},\hat{P}_Z^{CM},\hat{J}_x,\hat{J}_y,\hat{J}_z,\hat{p}_1,\hat{p}_2, \cdots
,\hat{p}_{3N-6} \right)
\end{equation}
where $\hat{P}_F^{CM}$ $(F=X,Y,Z)$ is the momentum conjugate to (associated with) the translation motion of the centre
of mass coordinate $R_F^{CM}$, ($\hat{J}_x, \hat{J}_y, \hat{J}_z$) are the $xyz$ components of the total angular momentum
and $\hat{p}_n = -i \hbar \partial / \partial \xi_n (n=1, \cdots , 3N-6)$ is the momentum conjugate to $\xi_n$. 

It is possible to write $\hat{T}$ in these generalised momenta as 
\begin{align}
\label{eq.general_T}
\hat{T} = & \frac{1}{2} \sum_{F=X,Y,Z} \hat{P}_F^{CM} G_{FF} \hat{P}_F^{CM} \\ \nonumber
          & + \frac{1}{2} \sum_{\alpha=x,y,z} \sum_{\alpha'=x,y,z} \hat{J}_{\alpha} G_{\alpha,\alpha'}(\xi) \hat{J}_{\alpha'} \\ \nonumber
          & -\frac{i \hbar}{2} \sum_{\alpha=x,y,z} \sum_{n=1}^{3N-6} \left[\hat{J}_{\alpha} G_{\alpha,n}(\xi) 
          \frac{\partial}{\partial \xi_n} + \frac{\partial}{\partial \xi_n} G_{\alpha,n}(\xi) \hat{J}_{\alpha} \right] \\ \nonumber
          & -\frac{\hbar^2}{2} \sum_{n=1}^{3N-6} \sum_{n'=1}^{3N-6} \frac{\partial}{\partial \xi_n} G_{n,n'}(\xi) 
          \frac{\partial}{\partial \xi_{n'}} + U(\xi).
\end{align}
This equation expresses the fact that the kinetic energy operator $\hat{T}$ can be expressed in terms of an expansion of the
generalised momenta with suitable `expansion coefficients' $G_{\lambda,\lambda'}$. 
The first term is the translation kinetic energy of the
centre of mass for which $G_{XX} = G_{YY} = G_{ZZ} = 1 / \sum_{j=1}^N m_j$. This term is exactly separable from the other terms
as expected. The second term is the kinetic energy of rotation, third term is the coupling between rotational and vibrational
motion, fourth term is the kinetic energy of vibrational motion and the final term is the pseudopotential term. For these 
terms all of the $G_{\lambda,\lambda'}$ depend on the complete set of vibrational coordinates $\xi$.  We can write 
equation \ref{eq.general_T} in the compact form 
\begin{equation}
\label{eq.general_T_compact}
\hat{T} = \frac{1}{2} \sum_{\lambda=1}^{3N} \sum_{\lambda'=1}^{3N} \hat{\Pi}_{\lambda} G_{\lambda,\lambda'}(\xi)
\hat{\Pi}_{\lambda'} + U(\xi) 
\end{equation}
where $\Pi_{\lambda}$ is an element of $\hat{\Pi}$ of equation \ref{eq.gen_momenta}. 

The vibrational coordinates $\xi_n$ can be any coordinates which represent the internal degrees of freedom and 
unambiguously define the instantaneous relative positions of the nuclei. Examples are internal displacement coordinates 
(i.e. displacement of bond lengths, angles and dihedral angles from equilibrium values), linearised interal coordinates
(see below) and symmetric combinations of these. This ability to choose which coordinates to use is the power of this approach
which makes it applicable to a wide variety of molecules. 

To utilise equation \ref{eq.general_T} the expansion terms $G_{\lambda,\lambda'}(\xi)$, pseudopotential term $U(\xi)$ and
the Born-Oppenheimer potential energy function $V$ must be expressed in terms of $\xi_n$. This is done by expressing these
quantities as a series expansion in terms of the $\xi$ themselves or functions of them
\begin{equation}
\label{eq.func_of_xi}
g_n = g_n(\xi_n).
\end{equation}
Thus, we can write
\begin{equation}
\label{eq.G_expansion}
G_{\lambda,\lambda'} = \sum_{l_1,l_2,l_3,\cdots} G_{l_1,l_2,l_3,\cdots}^{\lambda,\lambda'} g_1^{l_1} g_2^{l_2} g_3^{l_3} \cdots 
\end{equation}
and 
\begin{equation}
\label{eq.U_expansion}
U = \sum_{l_1,l_2,l_3,\cdots} U_{l_1,l_2,l_3,\cdots}^{\lambda,\lambda'} g_1^{l_1} g_2^{l_2} g_3^{l_3} \cdots 
\end{equation}
where $G_{l_1,l_2,l_3,\cdots}^{\lambda,\lambda'}$ and $U_{l_1,l_2,l_3,\cdots}^{\lambda,\lambda'}$ are constant expansion
coefficients. Similarly the potential $V$ is expressed as
\begin{equation}
\label{eq.V_expansion}
V = \sum_{l_1,l_2,l_3,\cdots} V_{l_1,l_2,l_3,\cdots} f_1^{l_1} f_2^{l_2} f_3^{l_3} \cdots
\end{equation}
where $V_{l_1,l_2,l_3}$ are constant expansion coefficients in terms of convenient expansion functions
\begin{equation}
\label{eq.v_exp_func}
f_n = f_n(\xi_n).
\end{equation}
For example $f_n = 1 - \exp(-a \xi_n)$ (Morse type) or $f_n = \cos(\xi_n)$. Typically Morse or Harmonic functions are used 
for bond stretches and $\xi_n$ is used itself for bends. 

The method of actually finding the expansion coefficients introduced above will now be discussed. This is arguably the most
technical part of the TROVE approach and could be skipped on first (or even second!) reading. It is based on a paper by
Sorensen \cite{79Soxxxx.method}.

To go from the expression for the kinetic energy in equation \ref{eq.schrodiger_lab_cart} to that in equation
\ref{eq.general_T} we start by noting that $\hat{T}$ in the former equation can be expressed as 
\begin{equation}
\label{eq.T_as_P}
\hat{T} = -\frac{\hbar^2}{2} \sum_{i=1}^N \frac{1}{m_i} \nabla^2_i = \sum_{X,Y,Z} \sum_{i=1}^{N} 
\frac{\hat{P}^2_{iF}}{2m_i} = \sum_{i=1}^N \frac{\hat{\mathbf{P}}_i^2}{2m_i}
\end{equation}
where the momentum vector $\hat{\mathbf{P}}_{iF}$ has the $XYZ$ coordinates ($\hat{P}_{iX}, \hat{P}_{iY}, \hat{P}_{iZ}$).
The chain-rule transformation in Hermitian form is defined as
\begin{equation}
\label{eq.chain_hermit}
\hat{P}_{iF} = \frac{1}{2} \sum_{\lambda = 1}^{3N} \left( s_{\lambda,iF} \hat{\Pi}_{\lambda} + \hat{\Pi}_{\lambda} 
s_{\lambda,iF} \right) 
\end{equation}
with
\begin{equation}
\label{eq.def_s}
s_{\lambda,iF} = \frac{\partial \Xi_{\lambda} }{\partial R_{iF} }.
\end{equation}
This relation states that the momentum in the $XYZ$ lab-fixed coordinate system $\hat{P}_{iF}$ can be expressed in terms
of the generalised momenta $\hat{\Pi}$ with the derivative of the generalised coordinates $\Xi$ with respect to a given 
lab-fixed coordinate $R_{iF}$ linking them. The Jacobian-matrix elements $s_{\lambda,iF}$ ($F = X,Y,Z$) define vectors and so
the vector from of equation \ref{eq.chain_hermit} is
\begin{equation}
\label{eq.chain_hermit_vec}
\hat{\mathbf{P}}_i = \frac{1}{2} \sum_{\lambda = 1}^{3N} \left(\mathbf{s}_{\lambda,i} \hat{\Pi}_{\lambda} +
\hat{\Pi}_{\lambda} \mathbf{s}_{\lambda,i} \right). 
\end{equation} 

When equation \ref{eq.chain_hermit_vec} is inserted into equation \ref{eq.T_as_P} the following equations for the 
$G_{\lambda,\lambda'}$ coefficients and pseudopotential term $U$ are given
\begin{equation}
\label{eq.G_with_s}
G_{\lambda,\lambda'} = \sum_{i=1}^N \frac{\mathbf{s}_{\lambda,i} \mathbf{s}_{\lambda',i}}{m_i}
\end{equation}
\begin{equation}
\label{eq.U_with_s}
U = \sum_{\lambda=1}^{3N} \sum_{\lambda'=1}^{3N} \sum_{i=1}^N \left\{  \frac{1}{8m_i} \left[\hat{\Pi}_{\lambda},
\mathbf{s}_{\lambda,i} \right] \cdot \left[\hat{\Pi}_{\lambda'},\mathbf{s}_{\lambda',i} \right] 
+ \frac{1}{4 m_i} \mathbf{s}_{\lambda,i} \cdot \left[\hat{\Pi}_{\lambda}, \left[\hat{\Pi}_{\lambda'},\mathbf{s}_{\lambda',i}
\right] \right] \right \}
\end{equation}
where the square brackets indicate the communicator of the quantities in them. 

To make progress the quantity $t_{iF,\lambda}$ is introduced with the definition
\begin{equation}
\label{eq.def_t}
t_{iF,\lambda} = \frac{\partial R_{iF}}{\partial \Xi_{\lambda}}.
\end{equation}
From the application of the chain rule the following relation is found
\begin{equation}
\label{eq.chain_s_t}
\sum_{i=1}^{N} \sum_{F=X,Y,Z} \frac{\partial \Xi_{\lambda} }{\partial R_{iF} } \frac{\partial R_{iF}}{\partial \Xi_{\lambda'}}
= \mathbf{s}_{\lambda,i} \cdot \mathbf{t}_{i,\lambda'} = \delta_{\lambda,\lambda'}
\end{equation}
where the vector $\mathbf{t}_{i,\lambda'}$ has been introduced. If the $\mathbf{t}_{i,\lambda'}$ vectors are known then
we can solve this equation to obtain the $\mathbf{s}_{i,\lambda'}$ vectors.

At this point further technical details of how to solve equation \ref{eq.chain_s_t} will not be given and instead the 
interested reader is referred to the TROVE paper \cite{TROVE} for more information. Instead a qualitative description
will be given. 

Sorensen \cite{79Soxxxx.method} showed what values the various components of the $\mathbf{t}_{i,\lambda'}$ vectors have,
consistent with Eckart conditions, which achieve optimum separation of rotational and vibrational motion. Equation
\ref{eq.chain_s_t} can then be solved numerically. Components of the $\mathbf{s}_{\lambda,i}$ and $\mathbf{t}_{i,\lambda'}$
are expanded as a power series in $g_n({\xi_n})$ (from equation \ref{eq.func_of_xi} above) to a given order
(this is what the integer after \verb|kinetic| refers to in the TROVE input file). When these 
power series are substituted into equation \ref{eq.chain_s_t} and coefficients up to a given order are collected, a
system of linear equations is obtained of form $\mathbf{T}\mathbf{x} = \mathbf{b}$. The systems of equations can be 
set up and solved numerically by making use of the fact that values of $\mathbf{t}_{i,\lambda'}$ are known.

The result of all this is that equations for $G_{\lambda,\lambda'}$ and $U$ given in equations \ref{eq.G_with_s} and
\ref{eq.U_with_s} are expressed in terms of products of $g_n(\xi_n)$ raised to powers and multiplied by expansion 
coefficients which are found from the linear equations described. This ultimately means that we can write $\hat{T}$ 
in terms of molecule-fixed $xyz$ coordinates as in equation \ref{eq.general_T}. The entire procedure
(although complicated) is a numerical one
and thus does not require any analytic algebra to define the kinetic energy operator for a given molecular shape. This is
what makes TROVE general. 


\section{Vibrational Coordinates}
The procedure described in the previous section for the numerical construction of the kinetic energy operator is general
and can be used with any choice of suitable vibrational coordinates $\xi_n$ as long as $t_{i \alpha,\mu}$ can be provided.
There are three basic types of coordinates used by TROVE: linearized coordinates, geometrically defined coordinates and
coordinates for non-rigid molecules with large amplitude vibrations. Of these, linearized coordinates tend to be 
used the most but geometrically defined coordinates have been used more recently due to a better implementation for them 
\cite{15YaYuxx.method}. Each type of coordinate shall be described in the next subsections.

\subsection{Linearized Coordinates}
The linearized coordinates are introduced in terms of the Cartesian displacements $d_{i \alpha}$ (where $i = 1$ to $N$ 
nuclei and
$\alpha = x,y,z$) of the nuclei from their equilibrium positions $a_{i \alpha}$ in the $xyz$ molecule-fixed axis system
\begin{equation}
\label{eq.linearized_def}
R^{MS}_{i \alpha} = a_{i \alpha} + d_{i \alpha}.
\end{equation}
In general the $3N - 6$ internal displacement coordinates $\xi_n$ are non-linear functions of the displacements $d_{i,\alpha}$
since, for example a bond stretch or bend will not usually lie along an axis. A set of $3N-6$ linearized coordinates
$\xi_n \equiv \xi_n^l$ are defined to be linear combinations of $d_{i \alpha}$ and to coincide with the $3N-6$ coordinates
$\xi_n$ in the linear approximation
\begin{equation}
\label{eq.linearized_def2}
\xi_n^l = \sum_{i=1}^N \sum_{\alpha=x,y,z} B_{n,i \alpha} d_{i \alpha}
\end{equation}
where $B_{n,i \alpha} = \partial \xi_n / \partial d_{i \alpha}$ are derived at equilibrium. The $B_{n,i \alpha}$ can be 
obtained from geometrical considerations (for example using trigonometry, etc).

The $xyz$ coordinate system has its origin at the molecule's centre of mass and so the constant equilibrium coordinates 
$a_{i \alpha}$ in equation \ref{eq.linearized_def} satisfy
\begin{equation}
\label{eq.centre_of_mass}
\sum_{i=1}^N m_i a_{i \alpha} = 0.
\end{equation}
The $a_{i \alpha}$ are easy to determine from the molecule's equilibrium geometry but they can be obtained numerically from the
Z-matrix. This gives an arbitrary molecule fixed axis $x'y'z'$ which is transformed to the principle axis system $xyz$ by
means of a diagonalization of the inertial matrix.

For linear coordinates the expansions needed for determining the kinetic energy operator are linear. This makes them 
amenable to be numerically solved. The details are given in the TROVE publication \cite{TROVE}. 
The simple form of the kinetic energy operator is an advantage of these coordinates.

\subsection{Geometrically Defined Coordinates}
Although linearized coordinates give a simple form for the kinetic energy operator they are not as good for expanding the 
potential energy. Geometrically defined coordinates have the advantage that when used, lower expansion orders are required for
an accurate representation of the potential. Geometrically defined coordinates are any convenient coordinates used to 
unambiguously define a molecule's geometry: for example, the bond lengths and angles from a Z-matrix.

A disadvantage of these coordinates is that the kinetic energy operator is 
harder to derive with the expansion being non-linear. The original TROVE publication describes how this can be carried out
numerically using `quadruple precision' in the program to calculate numerical derivatives accurately. 

A new way to obtain the expansion of the Hamiltonian was developed by Andrey Yachmenev by using `automatic differentiation'.
This is a computational method of obtaining derivatives of functions with the accuracy of symbolic algebra but carried
out in a numerical manner. The technical details of expanding the Hamiltonian and making use of the Eckart frame are 
discussed in detail in the publication \cite{15YaYuxx.method}. 
Examples comparing linear and geometrically defined (or `curvilinear') coordinates are also presented.

 


\subsection{Coordinates for Large Amplitude Vibrations}

If the kinetic and potential energy operators cannot be expanded in a Taylor series then a different approach is required.
This is the case for molecules with a large amplitude degree of freedom for example inversion in ammonia or torsional motion
in ethane. This degree of freedom will be labelled as coordinate $\rho$.

The method TROVE uses to handle this case is the Hougen-Bunker-Johns or HBJ approach. A grid of equidistant values along
$\rho$ is introduced. Each point of this grid is called a reference configuration. The remaining $3N-7$ small amplitude 
vibrational coordinates are then defined as displacements from this configuration. At each grid point along $\rho$ 
all relevant functions are expanded in terms of the small amplitude coordinates $\xi_n$. The steps given 
above for expanding the kinetic energy operator in either linearized or geometrically defined coordinates are carried out
at each grid point along $\rho$. The details are given in the TROVE paper \cite{TROVE}. 


\section{Expansion of the Potential Energy Function}

The potential energy function for a molecule is typically expressed in some suitable coordinates, ideally in a symmetrised 
form. This function is required as an input to TROVE (see chapter \ref{chap:newmol}) but for computational efficiency, 
TROVE re-expresses the potential in terms of the chosen coordinates $\xi$ (\ref{eq.v_exp_func})
\begin{align}
\label{eq.V_expand}
V(\xi_n) & =  \sum_{l_1 = 0}^L \sum_{l_2 = 0}^{(L-l_1)} \cdots \sum_{l_{(3N-6)-1}=0}^{ (L-l_1 \cdots l_{(3N-6)-2})} 
                 V_{l_1 l_2 \cdots l_{(3N-6)}}^L \prod_i f_n^{l_i}
         &  = \sum_{L=0}^{N_{pot}} \sum_{L[l]} V_{L[l]}(f_n)^{L[l]}.
\end{align}
This is a sum of products of the coordinates (or functions of the coordinates) used raised to powers. This 
means that all integrals involving the potential will be separable into products of one-dimensional integrals. 
The expansion coefficients are
obtained from the input potential using finite difference methods. This step also requires use of quadruple precision numbers
in the program to avoid the accumulation of round off errors. The order to expand the potential to, $N_{pot}$ is controlled by
the  \verb|potential| keyword in the TROVE input file.


\section{Vibrational Basis Functions and Matrix Elements}
\label{sec.Vib_basis_matelem}
TROVE solves the Schr\"{o}dinger equation using the variational method. This requires a suitable choice of basis 
functions for the method to be efficient. TROVE builds basis functions, starting from one-dimensional basis sets for
each vibrational motion. These are then combined and truncated to build up a basis for the full dimensionality of the 
molecule. The details of this process are given here.

From the previous sections the rotation-vibration Hamiltonian expanded in terms of molecule-fixed $xyz$ coordinates is given
(in notation introduced in equation \ref{eq.V_expand}) as
\begin{equation}
\label{eq.rovibH}
\hat{H}_{rv} = \frac{1}{2} \sum_{L \geq 0} \sum_{L[l]} \sum_{\lambda,\lambda'} \hat{\Pi}_{\lambda} G_{L[l]}^{\lambda,\lambda'}
(g)^{L[l]} \hat{\Pi}_{\lambda'} + \sum_{L \geq 0} \sum_{L[l]} U_{L[l]}(g)^{L[l]} 
+ \sum_{L \geq 0} \sum_{L[l]} V_{L[l]} (f)^{L[l]}
\end{equation}
with $g_n(\xi_n)$ and $f_n(\xi_n)$ defined in equations \ref{eq.func_of_xi} and \ref{eq.v_exp_func}. TROVE uses 
vibrational basis set functions $|\nu \rangle$ constructed as products of 1D basis functions
\begin{equation}
\label{eq.vib_basis_prod}
|\nu \rangle = \prod_{v} | \nu_v \rangle = \phi_{\nu_1}(\xi_1)\phi_{\nu_2}(\xi_2)\cdots \phi_{\nu_{3N-6}}(\xi_{3N-6}).
\end{equation}
The 1D basis functions implemented in TROVE are either analytically defined harmonic-oscillator or Morse-oscillator 
functions or are numerical solutions to the 1D Schro\"{o}dinger equations for each vibrational coordinate obtained
using  Numerov-Cooley integration. These numerical solutions are obtained by solving
\begin{equation}
\label{eq.1Dschrodinger}
\hat{H}_n^{(1D)} | \nu_n \rangle = E_{\nu_n} | \nu_n \rangle
\end{equation}
for the Hamiltonian
\begin{equation}
\label{eq.1D_Ham}
\hat{H}_n = -\frac{\hbar^2}{2} \frac{\partial}{\partial \xi_n} G_{n,n}^{(1D)}(\xi_n) \frac{\partial}{\partial \xi_n}
+ V^{(1D)}(\xi_n) + U^{(1D)}(\xi_n)
\end{equation}
where the other $3N-7$ coordinates are constrained to their equilibrium values to give $G_{n,n}^{(1D)}(\xi_n)$, 
 $V^{(1D)}(\xi_n)$ and $U^{(1D)}(\xi_n)$. 
 
 The vibrational matrix elements of the Hamiltonian in equation \ref{eq.rovibH} can all be expressed in terms of 
 one-dimensional integrals of each $\xi_n$ coordinate as
 \begin{align}
 \label{eq.1d_matrix_elem}
& V_{\nu_n,\nu'_n}^l(n) = \left< \nu_n | f_n^l(\xi_n) | \nu'_n \right>,\\
& T^{(0),l}_{\nu_n,\nu'_n}(n) = \left< \nu_n | g_n^l(\xi_n) | \nu'_n \right>, \\ 
& T^{(1),l}_{\nu_n,\nu'_n}(n) = \left< \nu_n | g_n^l(\xi_n) \frac{\partial}{\partial \xi_n} | \nu'_n \right>,\\
& T^{(2),l}_{\nu_n,\nu'_n}(n) = \left< \nu_n | \frac{\partial}{\partial \xi_n} g_n^l(\xi_n) \frac{\partial}{\partial \xi_n} |
  \nu'_n \right>.
 \end{align}
The integrals are computed in TROVE using Simpson's rule if numerically obtained basis functions are used or 
analytically if Harmonic or Morse oscillator functions are used. First derivatives are computed numerically using finite 
difference methods. Vibrational matrix elements of the Hamiltonian in \ref{eq.rovibH} are then given by products of the
matrix elements given in equations \ref{eq.1d_matrix_elem}. If the HBJ approach is required then these 1D matrix elements
are computed for each grid point along $\rho$ (see the TROVE paper \cite{TROVE} ). 

\section{Rotational Basis Functions}
\label{sec.rot_basis}
TROVE uses linear combinations of rigid-rotor functions given as linear combinations $|J,K,m,\pm \rangle$
\begin{align}
\label{eq.rigid_rot}
&|J,0,m,+ \rangle = |J,0,m \rangle, \\
&|J,K,m,\pm \rangle = \frac{p(J,K,\pm)}{\sqrt{2}} \left(|J,K,m\rangle \pm |J,-K,m\rangle \right)
\end{align}
where $J$ is the total angular momentum (specified by the \verb|0,'JKtau', Jrot n| part of the TROVE input file in the 
basis block), $K$ and $m$ are projections of $J$ onto a certain axis. $\frac{p(J,K,\pm)}{\sqrt{2}}$ is a phase factor
chosen to make the matrix representations of the kinetic energy operator real. 

Descriptions of these functions are given in introductory textbooks to quantum mechanics \cite{11Atkins.book} and in detail in
Bunker and Jensen's book \cite{98BuJexx}. 
Matrix elements of these functions with the $\hat{J}_{\alpha}$ operators are analytical.

The complete basis set which to be used in TROVE was a combination of these functions with the vibrational functions
\begin{equation}
\label{eq.rovib_basis}
|\nu,J,K,m,\pm \rangle = \prod_{v} |\nu _v \rangle \times |J,K,m,\pm \rangle.
\end{equation}
This form of basis set can still be used in TROVE but it is much efficient to use the `$J=0$' method discussed below.


\section{Diagonalisation of the Hamiltonian}

The previous sections of this chapter have described: how the rotational-vibrational Hamiltonian is expanded in terms 
of internal coordinates of the molecule, the vibrational basis functions used in TROVE and how matrix elements of them
are computed and the rotational basis functions used in TROVE. With all of this in place, the final computation required
to obtain the rotational-vibrational energies and eigenfunctions is to diagonalise the Hamiltonian matrix.

The Schr\={o}dinger equation in matrix form is written as 
\begin{equation}
\label{eq.Schrodinger_matrix}
\mathbf{H}\mathbf{C} = \mathbf{E}\mathbf{C}
\end{equation}
where $\mathbf{H}$ is the Hamiltonian matrix, $\mathbf{C}$ is a matrix of coefficients and $\mathbf{E}$ is a diagonal
matrix of energies (or `eigenvalues'). $\mathbf{H}$ contains matrix elements of \ref{eq.rovibH} with the basis functions
of equation \ref{eq.rovib_basis}. $\mathbf{C}$ is a matrix of (unknown) coefficients which multiply each basis function
of equation \ref{eq.rovib_basis} to give a variational approximation to the eigenfunction of that rotational-vibrational state.
 Each column will give the coefficients required for a single state. $\mathbf{E}$ contains the energies of each state. Equation
\ref{eq.Schrodinger_matrix} is an eigenvalue equation. To solve it the Hamiltonian matrix is `diagonalised'. This is a 
standard problem in many areas of science and mathematics and general programs have been written for its solution. TROVE
uses the LAPACK/BLAS libraries. The full Hamiltonian decouples into blocks of independent $J$ and symmetry $\Gamma$ that is, matrix elements between different $J$s and $\Gamma$s are zero. This greatly reduces the size of the matrices to 
be diagonalised.

After diagonalisation of $\mathbf{H}$ the coefficients are stored (if \verb|Eigenfunc SAVE| is used). Further calculations
using the eigenfunctions (for example, obtaining transition intensities) are then simplified into multiplying and adding
the corresponding coefficients together and multiplying pre-computed integrals. 




\section{Symmetrised Basis Functions in TROVE}

Symmetry plays a crucial part in the TROVE program and the calculation of molecular energy levels and spectra in general. 
Using symmetry systematically via the application of Group Theory \cite{11Atkins.book} can greatly reduce the effort required
to solve the Schr\={o}dinger equation as many of the required matrix elements which are zero can be shown to be so without
computing them explicitly. Symmetry is also required to assess which spectroscopic transitions are possible \cite{98BuJexx}..

TROVE implements symmetry methods in a numerical manner. The following section is based on a recent paper by 
Yurchenko, Yachmenev and Ovsyannikov \cite{17YuYaOv.methods} 
which discusses TROVE's implementation of symmetry in a pedagogical manner
with examples. The reader is referred there for more detail and only a summary is given here. 

Following the symmetry paper the rotational-vibrational basis functions of equation \ref{eq.rovib_basis} are written as
\begin{equation}
\label{eq.rovib_basis2}
\Phi_{k,\nu}^J(\theta,\phi,\chi,\xi_1,\xi_2\cdots, \xi_{3N-6}) = \prod_{v} |\nu_v \rangle \times |J,K,m,\pm \rangle.
\end{equation}
Symmetry adapted basis functions are formed from linear combinations of these primitive functions as
\begin{equation}
\label{eq.sym_adapted_basis}
\Psi_{\mu,n}^{J,\Gamma_s} = \sum_{k,v} T_{k,v,n}^{\mu,J,\Gamma_s} \Phi_{k,\nu}^J.
\end{equation}
In this equation the $T_{k,v,n}^{\mu,J,\Gamma_s}$ are symmetrization coefficients (not to be confused with the
variational expansion coefficients of equation \ref{eq.Schrodinger_matrix}). Here $\mu$ is a counting number,
$\Gamma_s$ is symmetry label of a certain irreducible representation (irrep) of the symmetry group (see Atkin's MQM for 
a good introduction to this \cite{11Atkins.book}) and $n$ is used for degenerate symmetries.

Symmetrised basis functions have the important advantage that they the make the Hamiltonian block diagonal. That is
\begin{equation}
\label{eq.Ham_block_diag}
\left< \Psi_{\mu,n}^{J,\Gamma_s} | H^{rv} | \Psi_{\mu',n'}^{J,\Gamma_t} \right>  = H_{\mu,\mu'} \delta_{s,t}\delta_{n,n'}
\end{equation}
so that each $J_{\Gamma_s,n}$ Hamiltonian block can be diagonalised independently. This gives a huge time and memory
saving, especially for large basis sets and allows the calculation of different symmetries to be carried out in
parallel. It also means that $J$, $\Gamma_s$ (and $n$ a symmetry label for degenerate states) 
can be considered `good' quantum numbers for labelling states. 
With the advantage of symmetrised functions noted, the method for obtaining them used in TROVE will be described.

The Hamiltonian operator for a system $\hat{H}$ commutes with all operations of a given symmetry operation $R$
\begin{equation}
\label{eq.Ham_commute}
\left[\hat{H},R\right] = 0
\end{equation}
and eigenfunctions of $\hat{H}$ are also eigenfunctions of $R$ (as a simple example of this, a hydrogen s-orbital is invariant
under all operations of the spherical group $R^3$). This means that the eigenfunctions transform as an 
irrep of the symmetry group, $\mathbf{G}$. 

The full rovibrational Hamiltonian $H^{rv}$ is not used to find symmetrised functions since this is exactly the process we
are trying to simplify. Instead a set of reduced Hamiltonians $\hat{H}^{(i)}$ is introduced, similar to what was done 
for finding 1D basis functions in equation \ref{eq.1Dschrodinger}. The approach used in TROVE for this is as follows:

(i) All ro-vibrational degrees of freedom are divided into $L$ symmetrically independent subspaces which form subgroups of 
$\mathbf{G}$. For example in the PF$_3$ example from chapter \ref{chap:Quickstart}, the basis block was divided into `1s' 
and `2s' for the stretches and bends respectively.

(ii) For each subspace $i = 1, \cdots, L$, a reduced Hamiltonian operator $\hat{H}^{(i)}$ is constructed by neglecting 
or integrating over the other degrees of freedom.

(iii) The symmetry-adapted wave functions for each subspace are obtained by diagonalising the corresponding $\hat{H}^{(i)}$.

(iv) The total basis set is built as a direct product of the subspace bases and transformed to irreps using standard approaches.

Symmetrically independent subspaces of coordinates are chosen such that each subspace contains only coordinates which can be
symmetrically related by operations of the symmetry group (for example the three stretches of PF$_3$ for one subspace and the
three bends as the other). 

The details of the above steps are as follows. For each subspace a reduced eigenvalue problem is given by
\begin{equation}
\label{eq.Schrodinger_subspace}
\hat{H}^{(i)}(\mathbf{Q}^{(i)})\Psi^{(i)}_{\lambda_i}(\mathbf{Q}^{(i)}) = E_{\lambda_i}\Psi^{(i)}_{\lambda_i}(\mathbf{Q}^{(i)})
\end{equation}
where $\mathbf{Q}^{(i)}$ is a set of coordinates ($\xi_1,\xi_2,\cdots$) from a subspace $i$ and $\lambda_i$ is a counter
of each solution from $i$. The eigenfunctions will transform as an irrpe of the molecular symmetry group $\mathbf{G}$. The 
reduced Hamiltonian is constructed by averaging the total vibrational ($J=0$) Hamiltonian $\hat{H}$ on the 
ground-state primitive vibrational basis functions of the other subspaces
\begin{equation}
\label{eq.reduced_H}
\hat{H}^{(i)}(\mathbf{Q}^{(i)}) = \left< 0_p| \langle 0_q | \cdots \left<0_r|\hat{H}|0_r \right> \cdots |0_q \rangle |0_p
 \right>
\end{equation}
As well as giving symmetrised functions, solving equation \ref{eq.Schrodinger_subspace} also gives better basis functions
for the system since the problem is closer to the full dimensionality. The solutions can also be contracted, by energy for
example. The TROVE symmetry paper gives examples of how the method works for AB$_2$ and XY$_3$ type molecules. 
The total basis set for the full dimensionality of the molecule is constructed by a direct product of the $L$
symmetrised basis sets. This is then transformed to irreps using standard approaches. 

Although the solutions of the reduced Schr\"odinger equations are guaranteed to be an irrep of the symmetry group $\mathbf{G}$ 
it may not be obvious to which symmetry
a given function belongs. Degenerate solutions will also be mixed together. TROVE solves both of these problems in a 
numerical manner. To determine which irrep a given solutions belongs to, TROVE samples the basis functions on a grid of
geometries $N^{(i)}_{\text{grid}}$. The number of these points used is the value of \verb|sample_points| in the TROVE input file.
For a given subspace $i$, a random grid of geometries of that space 
$\mathbf{Q}_k^{(i)}$($k=1,\cdots,N^{(i)}_{\text{grid}})$, all
symmetry related images $R (\mathbf{Q}^{(i)})$ are generated. These are used to find the values of the wave functions 
$\Psi^{(i)}_{\lambda_i}(R \mathbf{Q}^{(i)})$ at each geometry. This allows the transformation matrices $\mathbf{D}[R]$ for 
each operation of the group $\mathbf{G}$ to be established and the symmetry of wave functions to be worked out.

The same procedure is used to obtained symmetrised functions for $J>0$ rotational-vibrational states.


\section{The $J=0$ Contraction Method}
The basis functions described in section \ref{sec.rot_basis} which are a product of rigid-rotor and primitive 
(or symmetry-adapted) basis functions can in principle be used for $J>0$ calculations. This approach requires the full
 Hamiltonian matrix 
for each symmetry to be diagonalised each time and ignores the fact that the purely vibrational $J=0$ problem has already
been solved. A better approach is to use the $J=0$ vibrational solutions as a basis for $J>0$ calculations. This is the
$J=0$ contraction. 

The $J=0$ vibrational eigenfunctions $\Psi_{J=0,i}^{\Gamma_s}$ for each symmetry $\Gamma_s$ of the molecule is 
first obtained by diagonalising the vibrational Hamiltonian. These are then multiplied by the rigid rotor functions 
discussed in section \ref{sec.rot_basis} and symmetrised. This gives a basis $\Psi^{\Gamma_s}_{J,K,i}$.

From section \ref{sec.numerical_T} the Hamiltonian is given as 
\begin{equation}
\label{eq.general_H_simp}
\hat{T} =  \frac{1}{2} \sum_{\alpha,\alpha'} \hat{J}_{\alpha} G_{\alpha,\alpha'}(\xi) \hat{J}_{\alpha'} 
           -\frac{i \hbar}{2} \sum_{\alpha,n}  \left[\hat{J}_{\alpha} G_{\alpha,n}(\xi) 
          \frac{\partial}{\partial \xi_n} + \frac{\partial}{\partial \xi_n} G_{\alpha,n}(\xi) \hat{J}_{\alpha} \right] 
           + \hat{H}_{\text{vib}}
\end{equation}
where the centre of mass motion has been ignored and simplified notation used. Here $\hat{H}_{\text{vib}}$ is given as
\begin{equation}
\label{eq.Hvib}
\hat{H}_{\text{vib}} = -\frac{\hbar^2}{2} \sum_{n,n'}  \frac{\partial}{\partial \xi_n} G_{n,n'}(\xi) 
          \frac{\partial}{\partial \xi_{n'}} + U(\xi) + V.
\end{equation}
The functions $\Psi_{J=0,i}^{\Gamma_s}$ are solutions for this Hamiltonian and satisfy
\begin{equation}
\label{eq.vib_orth}
\left< \Psi_{J=0,i}^{\Gamma_s} | \hat{H}_{\text{vib}} | \Psi_{J=0,i'}^{\Gamma_s} \right> = E_i^{\text{vib}} \delta_{i,i'}.
\end{equation}

Calculating matrix elements of the Hamiltonian equation \ref{eq.general_H_simp} can be further simplified by pre-computing
integrals using the $J=0$ basis
\begin{equation}
G_{\alpha,\alpha'}^{\Gamma_s,\Gamma_s',i,i'} = \left< \Psi_{J=0,i}^{\Gamma_s} | G_{\alpha,\alpha'} | \Psi_{J=0,i'}^{\Gamma_s'}
\right>
\end{equation}
and
\begin{equation}
G_{\alpha,n}^{\Gamma_s,\Gamma_s',i,i'} = \left< \Psi_{J=0,i}^{\Gamma_s} | \left[\hat{J}_{\alpha} G_{\alpha,n}(\xi) 
          \frac{\partial}{\partial \xi_n} + \frac{\partial}{\partial \xi_n} G_{\alpha,n}(\xi) \hat{J}_{\alpha} \right]
          | \Psi_{J=0,i'}^{\Gamma_s'} \right>.
\end{equation}
Matrix elements are neglected if the values are below a certain tolerance, usually 10$^{-16}$. This is the last step where
the primitive basis set is required. Many of the matrix elements involving the rigid-rotor functions are analytic. 

The $J=0$ contraction greatly speeds up the calculation of $J>0$ matrix elements. Matrix elements of the dipole moment surface
can also be calculated using a similar approach.

Another feature of this approach is the possibility to use experimental band centres in equation \ref{eq.vib_orth} instead
of calculated vibrational energies. This is denoted the `empirical basis set correction' since effectively the vibrational
basis set is improved (there is no correction to the rotational structure using this method). This is a useful and pragmatic
approach when many experimental energies are available, especially if the band of interest has a Q-branch. Even after
refinement some bands may not agree satisfactorily and so can be corrected using this method. In TROVE this is implemented
by changing the values in the j0descr.chk files.




\section{Intensity Calculations in TROVE}

Transition intensities can be calculated using TROVE but for the production of line lists, the GAIN program is recommended.
To calculate intensities a dipole moment surface (DMS) for the molecule of interest is required. This is similar to a PES
but instead of giving the molecule's electronic energy as a function of molecular geometry, it gives a molecule's dipole. 
Since this is a vector quantity a DMS has three values associated with a given molecular geometry: one for each X,Y,Z
coordinate. 

Similar to the PES, TROVE expands the DMS in terms of internal coordinates of the molecule to a given expansion order chosen
by the user. Matrix elements of the DMS between basis functions are computed in TROVE and can also be converted to the 
$J=0$ contraction scheme for use in $J>0$ calculations. The pre-computation of these matrix elements allows for faster
computation of transition intensities involving eigenfunction of each ro-vibrational state.

The Einstein-A coefficient for a particular transition from the initial state $i$ to the final state $f$ is given by
\begin{equation}
\label{eq.einsteinA}
A_{if} = \frac{8 \pi^4 \nu^3_{if}}{3h} (2J_i + 1) \sum_{\alpha = x, y, z} \left|
 \bra{\Psi^f}  \bar{\mu}_{\alpha} \ket{\Psi^i}  \right| ^2
\end{equation}
where $J_i$ is the rotation quantum number for the initial state, $h$ is Planck's constant, $\nu_{if}$ is the
transition frequency ($hc \cdot \nu_{if} = E_f - E_i$) and $\Psi^f$ and $\Psi^i$ are the initial and final rovibrational states
respectively. Since matrix elements of the dipole between states are pre-computed by TROVE this integral becomes a sum
of terms. Technical details of how these integrals are evaluated is given in the GAIN paper \cite{GAIN}. 

The Einstein-A coefficients are costly to compute but note that they are temperature independent. Once computed for transitions
between all states of interest (usually to some value of $J$), the transition intensities (and spectra) 
for any temperature can be computed relatively straightforwardly (using Exocross \cite{Exocross} for example). 

The absolute absorption intensities are given by 
\begin{equation}
\label{eq.intensity}
I(f \leftarrow i) = \frac{A_{if}}{8 \pi c} g_{ns} (2 J_f + 1) \frac{\exp(-E_i/kT) }{Q(T) \nu^2_{if}}
\times \left[ 1 - \exp\left( - \frac{c_2 \nu_{if}}{T} \right) \right]
\end{equation}
where $k$ is the Boltzmann constant, $T$ is the absolute temperature, $Q(T)$ is the partition function, $g_{ns}$ is the 
nuclear statistical weight and $c_2 = hc/k$. 


















