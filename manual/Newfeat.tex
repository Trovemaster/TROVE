
\chapter{New Features}
\label{chap:newfeat}

This chapter discusses the latest additions to TROVE and their implementation. Many of these new features are still being 
tested and developed and so this chapter, even more than others, should be treated as subject to change. These features
have already proven important however and so will likely remain a part of TROVE but perhaps with a different implementation.

Most of these new features are for dealing with larger molecules in TROVE.




\section{Splitting and Stitching}

As discussed in Chapter \ref{chap:Quickstart}, matrix elements between contracted basis functions are saved to a file,
\verb|contr_matelem.chk|. This step is time consuming and for large molecules/basis sets it may not be possible to carry out
this step within a computer's allocated time limit. The \verb|split| command solves this problem by splitting the matrix 
into 12 `slices'. Each slice contains matrix elements of a component of the full Hamiltonian operator.

To use splitting, in the \verb|CHECK_POINT| block the \verb|matelem| should be set to
\begin{verse}
matelem  save  split i j 
\end{verse}
where \verb|i| and \verb|j| are integers between 0 and 12 (and \verb|j| $\geq$ \verb|i|). The most parallel way of saving 
the matelem files is to calculate each separately by setting \verb|i| and \verb|j| to 0 0, 1 1, ...12 12. Splitting can
also be used for the dipole using
\begin{verse}
extmatelem  save  split i j 
\end{verse}
where \verb|i| and \verb|j| are now 1,2 or 3 and should be run with 1 1, 2 2, 3 3.

Using splitting will produce files \verb|matelemI.chk| where \verb|i| is 1-12 and \verb|extmatelemJ.chk| where \verb|J| 
is 1-3. 

To convert these files to the `J=0' representation, use
\begin{verse}
matelem  convert  split 
\end{verse}
This will produce \verb|j0_matelemI.chk| files. This can be used in subsequent calculations using \verb|read| as usual,
as long as \verb|split| is specified also so that the keyword should be set to
\begin{verse}
matelem  read  split 
\end{verse}

Matrix elements of the dipole are converted in the same way. For the dipole it may be necessary to combine the split 
extmatelem files however using the \verb|stitch| option as currently the GAIN program does not accept split files.
To carry this out, after the three extmatelem files have been calculated, re-run TROVE with
\begin{verse}
extmatelem convert stitch
\end{verse}
This will combine the split matelem files into a \verb|j0_extfield.chk| files (which are produced if not splitting is 
carried out as specified in Chapter \ref{chap:Quickstart}). This can then be used with GAIN.



\section{Restarting}

For large molecules and basis sets it may be that the vibrational part of the calculation does not finish within a 
large computer's time limit. In this case it is usually the calculation of matrix elements which is the time consuming part.
Since these are independent calculations however, the elements can be saved and read back in to form the full Hamiltonian 
matrix.

To use this restart feature, in the \verb|CHECK_POINT| block the \verb|matelem| keyword should be set to
\begin{verbatim}
matelem  save  split  0 0 dump
\end{verbatim}
In this case the \verb|dump| keyword gets TROVE to sequentially save the matrix at intervals. This will produce an output 
file which after giving the dimensions of the contracted matrix will print out to which matrix element TROVE has saved.
If the calculation then runs out of time, the user can read back in these pre-computed elements up to that point by 
specifying
\begin{verbatim}
matelem  append n split  0 0 dump
\end{verbatim}
In this case TROVE restarts at \verb|n| and continue to calculate matrix elements. This can be repeated until the matrix 
is filled and be diagonalized. 

Again, this is only for the $J=0$ vibrational step and is useful when used in conjunction with the `J=0' step and 
especially with pruning methods, as described below.


\section{Fast-CI}

The Fast-CI method was implemented in TROVE by Andrey Yachmenev as a way of speeding up calculations. The method does this by
essentially re-ordering loops and indices in a more efficient manner. Another aspect of the method is the ability to 
choose a cut-off where expansion coefficients of the Hamiltonian are discarded if too small. The method is optimally designed
when using curvilinear coordinates.

To use the Fast-CI method, in the \verb|CONTRACTION| block put \verb|fast_ci|. To make use of the cut-off option put
\verb|exp_coeff_thresh| in the same block of input followed by what value is required. Typically very small values such as
$1\times10^{-7}$ or so should be used. This should result in essentially the same results as not using any cut-off
but with significant speed up. If even faster calculations are required the value should be increased but values
larger than $1\times10^{-2}$ are not recommended. Ideally this parameter should be benchmarked using smaller basis sets
to get an idea of the particular systems sensitivity to its value. 

The Fast-CI method generates different checkpoint files for the vibrational matrix elements compared to what is produced
using standard TROVE options. Many files starting with \verb|contrME_| are produced which have to do with the ordering
procedure. 

When using Fast-CI, TROVE can be used as normal for other steps in the calculation and the \verb|fast_ci| and 
\verb|exp_coeff_thresh| keywords can remain the the \verb|CONTRACTION| block.
 




\section{Storing Hamiltonian and External Diagonalization} 

For larger production calculations involving high values of $J$ it is likely that the construction and diagonalization of
the Hamiltonian matrix will exceed the time limit of the computer which is used. In this case the Hamiltonian matrix can be
saved and then diagonalized separately for each symmetry.

To save the Hamiltonian for a given symmetry, in the \verb|DIAGONALIZER| block put
\begin{verbatim}
save
gamma i
\end{verbatim}
where \verb|i| is the symmetry. This should be done for each symmetry at a given $J$. TROVE will produce the files 
\verb|matrixJ_I.chk| where \verb|J| is the value of $J$ and \verb|I| is the symmetry. TROVE will also produce 
eigenvector and descriptor files but these will be empty.

These matrices can then be diagonalized using an external program. An example of such a program is PDSYEVD, a ScaLapack
program which is MPI parallel and so can be run on multiple CPU nodes. A TROVE compatible driver program for this is 
available from Sergey Yurchenko. An example input for this program is 
\begin{verbatim}
(title)
J 32
gamma 8
DIAGONALIZER pdsyevd
ENERGY_THRESH 16000.0
COEFF_THRESH  1e-18
ZPE 11022.4701
MEM 64 gb
\end{verbatim}
where the keywords are the same as those used in TROVE input.

This program will produce the eigenvectors for the specified $J$ and $\Gamma$ as TROVE would but not the descriptor files
(since only the Hamiltonian matrix was specified without details of the basis set, etc). The program also produces a
\verb|energiesJ_I.chk| file which contains the eigenvalues.

To produce the relevant descriptor files and usual TROVE output files, TROVE should be re-run with the \verb|energiesJ_I.chk|
file in the same directory with the keywords in the \verb|DIAGONALIZER| block changed to
\begin{verbatim}
read-energies
gamma i
\end{verbatim}
This will cause TROVE to read the energies file and produce the usual descriptor files and output block containing the 
energies, quantum numbers, etc. This is essentially a `bookkeeping' step and does not require much computing time or memory.


\section{Transition Moment Intensity Pruning}

Another new method which has been developed to reduce calculation time of line lists for large molecules is transition 
moment intensity pruning. This procedure reduces how many vibrational levels are included based on their intensity. Levels 
which have very weak intensities for both transitions to and from them are discarded. This results in a large reduction of the 
basis set but should only remove transitions of very low intensity. 

As the method prunes the basis using the vibrational intensities, it assumes that rotational levels with the same vibrational
quantum numbers will also be weak. 

To use this method, the usual steps for calculating the transition moment should be followed but the keyword
\verb|pruning| should be added into the intensity block. TROVE will then calculate the transition moments and intensities 
as usual but
also work out and store the most intense transitions to and from each state. This calculations produces the checkpoint files
\verb|eigen_intens0_n.chk| for each symmetry \verb|n|. In applications the temperature for this step has been set to 
around the maximum for which the line list being calculated to try and make sure no important states are left out.

The basis set can then be pruned using the `J=0' method. In the \verb|CONTRACTION| the following should be included
\begin{verbatim}
tm_cutoff  1e-24
tm_enermin 8000.0
\end{verbatim}
\verb|tm_cutoff| sets the minimum intensity for removing states. This should ideally by set to as low a value as possible and
will depend on practical considerations such as computing time and memory. \verb|tm_enermin| is the minimum energy in 
wavenumbers for which pruning will occur. In this example, all states below 8000 cm$^{-1}$ will be included in the 
basis set regardless of intensity. This value should be set as large as possible but will again be determined by practical
considerations.

An example of using this procedure is for the ethylene (C$_2$H$_6$) molecule \cite{jt729}. 
For this relatively large molecule a basis set
with a polyad number of 10 produced split \verb|matelem| files which were 158 Gb each. Using these basis sets would not
have been practical at high $J$s. Using the pruning method with the parameters as given in the example above reduced the 
matelems to 1.4 Gb. This then allowed refinement and a full line list calculation to be carried out.

