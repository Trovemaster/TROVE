

\chapter{Refinement}
\label{chap:refine}

In this chapter details will be given of the refinement procedure implemented in TROVE. 
Refinement in this context means to adjust the \textit{ab intio} potential energy 
surface by comparing computed rotational-vibrational energy levels to experimental values. 
Parameters of the PES are varied, energy levels re-computed and compared to experiment.
This process is continued until acceptable agreement between the calculated and experimental energy levels is obtained. 
Usually there is relatively few experimental energies
and so \textit{ab intio} electronic energies are used to constrain the refinement to prevent over fitting. 
Although experimental data is usually at fairly low energies, it is often the case that correcting the lower energy 
region of the PES gives more accurate values at high energies also. 

\section{Refinement with TROVE: Theory}
Details of the method of refinement implemented in TROVE have been published\cite{jt503} and only a brief summary 
will be given here. Assuming a reasonable PES has
already been obtained, a correction is added in terms of internal coordinates $\xi$
\begin{equation}
 \Delta V = \sum_{ijk...} \Delta f_{ijk...} \left(\xi_1^i \xi_2^j \xi_3^k ...\right)^A
\end{equation}
where $\left(\xi_1^i \xi_2^j \xi_3^k ... \right)^A$ corresponds to totally symmetric permutation of the internal coordinates 
so that all symmetry properties
of the molecule are properly accounted for. $\Delta f_{ijk}$ are the expansion coefficients which are found by refinement. 
The Hamiltonian is now given as
\begin{equation}
 H = T + V + \Delta V = H_0 + \sum_{ijk...} \Delta f_{ijk...} \left(\xi_1^i \xi_2^j \xi_3^k \right)^A
\end{equation}
where $H_0$ is the initial Hamiltonian with \textit{ab initio} PES. 

If the eigenvalue problem for the initial Hamiltonian has been solved,
\begin{equation}
 H_0 \psi^{J,\Gamma}_{0,i} = E^{J,\Gamma}_{0,i} \psi^{J,\Gamma}_{0,i}
\end{equation}
where $J$ is the total angular momentum quantum number and $\Gamma$ is a symmetry label, then matrix elements of $H$, 
using the $H_0$ solutions as a basis, are
\begin{equation}
 \left< \psi^{J,\Gamma}_{0,i} | H |\psi^{J,\Gamma}_{0,i'}   \right> = E^{J,\Gamma}_{0,i} + \sum_{ijk...} \Delta f_{ijk...} \Xi_{i,i'}^{J, \Gamma}
\end{equation}
where 
\begin{equation}
 \Xi_{i,i'}^{J, \Gamma} = \left< \psi^{J,\Gamma}_{0,i} | \left(\xi_1^i \xi_2^j \xi_3^k ...\right)^A | \psi^{J,\Gamma}_{0,i'} \right>.
\end{equation}

The derivatives of the energies with respect to adjustable parameters, which are requred for least squares fitting, 
are given by the Hellman-Feynman theorem\cite{11Atkins.book,jt503}
\begin{equation}
 \frac{\partial E^{J,\Gamma}_{n} }{ \partial \Delta f_{ijk...} } = \left< \psi^{J,\Gamma}_{n} \left| \frac{\partial \Delta V}{\partial \Delta f_{ijk...} } 
 \right |\psi^{J,\Gamma}_{n} \right> = \left< \psi^{J,\Gamma}_{n} \left| \left(\xi_1^i \xi_2^j \xi_3^k ...\right)^A \right| \psi^{J,\Gamma}_{n} \right>.
\end{equation}
where $E^{J,\Gamma}_{n}$ and $\psi^{J,\Gamma}_{n}$ are eigenvalues and eigenvectors of $H$ respectively. 
In the J=0 representation $\psi^{J,\Gamma}_{n}$ is given by
\begin{equation}
 \psi^{J,\Gamma}_{n} = \sum_i C_i^{J, \Gamma} \psi_{0,i}^{J, \gamma}
\end{equation} 

As the derivative of the energy levels with respect to the correction parameters are given, standard least squares fitting
procedures can then be used to determine how they should be varied. This is all implemented in TROVE.



\section{Refinement Implementation with TROVE}

\subsection{Setting up Refinement}

The specific inputs and checkpoint files required to carry out refinement of a PES using TROVE is discussed in this section

Prior to refinement TROVE requires checkpoint files and eigenfunctions for the basis set being used (see above). 
If a calculation of the rotational-vibrational levels using an
unrefined PES has already been carried out, then all necessary files for refinement will have been generated. 
Refinement can be carried out in the `J=0' basis.

The refinement parameters ($\Delta f_{ijk...}$ from previous section) are defined in the `external' block on the TROVE input
 file. If the PES to be refined is of the same form,
that is, in terms of a polynomial of symmetrised internal coordinates, then the refinement parameters will just be a repeat of 
the potential block. 

The required matrix elements of the refinement parameters are computed in stages. First matrix elements of the primitive 
basis functions used by TROVE are calculated.
This is carried out by putting 
\begin{verbatim}
 extmatelem save split n n
\end{verbatim}
in the TROVE input file in the checkpoint block. n is the number of a specific expansion parameter. 
If all parameters are required this can be ignored but this may not 
be possible for all molecules or if there are lots of expansion parameters. 
This will generate extmatelemn.chk files (where n is number of expansion parameter). 
These files can then be converted to the `J=0' representation using 
\begin{verbatim}
 extmatelem convert split n n.
\end{verbatim}

As discussed above, the refinement procedure requires matrix elements of the $H_0$ Hamiltonian and so eigenfunctions for 
each $J$ of interest must be computed. To save matrix elements 
of the eigenfunctions, TROVE is run for each $J$ with 
\begin{verbatim}
fit_poten save split
\end{verbatim}
in the checkpoint block. This generates fitpot-mat$J$-$\Gamma$-n.chk files. Since a file is 
generated for each $J$ and symmetry $\Gamma$ for each expansion parameter 
n, many files are generated in this step. 


\subsection{Running Refinement}

When all of the required checkpoint files described above have been generated, TROVE can be used to refine the PES. The following block should be added to the TROVE input file
\begin{verbatim}
FITTING
J-LIST         0 1......
symmetries     1 2 .....n
itmax          10
fit_factor     100
lock           20.0
output         fittest
robust         0.0
geometries     c2h4_pes.dat
OBS_ENERGIES   52 (J  symmetry NN Energy 12 QNs weight  )
0 1 2  1343.31   0   0   0   0   0   0   0   1   0   0   0   0   10  e  o
.
.
.
\end{verbatim}
\verb|J-LIST| is a list of total angular momentums to be included in the refinement, all checkpoint files for $J$'s 
selected must have been already computed. 

\verb|symmetries| is a list of
symmetries to be included, again all checkpoint files for each $\Gamma$ must have already been computed. 

\verb|itmax| is the number of iterations of refining carried out. 

\verb|fit_factor| is the relative weighting for the experimental data compared to \textit{ab initio} energies. 
The larger this is, the more importance will be given to the
experimental energies. 

\verb|output| is a string which specifies the pre-fix 
for output file names. 

\verb|robust| specifies whether Watson Robust
fitting is used, for 0.0 it is not, for 2.0 it is. 

\verb|geometries| is the name of the file which contains 
\textit{ab initio} energies. This file should give geometries in the
same coordinates as specified by the potential energy surface for the molecule of interest in TROVE 
followed by the \textit{ab initio} energy (from Molpro for example) and a weighting.

\verb|OBS_ENERGIES| is the number of observed (experimental) energies used. Below this a list of energies is given in the format
\begin{verbatim}
 J \Gamma NN E_i t_1 t_2 t_3 . . .    weight e o
\end{verbatim}
where $J$ and $\Gamma$ are the angular momentum and symmetry number of the energies, NN is the block number, 
which is the number of the energy given by TROVE. The following numbers
are the TROVE assignment of the energy level, followed by a weighting. 

With the fitting block added to the input, TROVE can be used to refine a PES. In the external block \verb|NPARAM| 
should be set to the number of parameters which are to be refined.
In the list of parameters, the first column of integers specifies if a parameter is to be refined. `1' 
will include in refinement, `0' will exclude. The next column of
real numbers are the starting values of the refinement parameters and should be set to 0.0 if initial refinement.

To carry out refinement all parts of the checkpoint block should be set to `read' or `none'. TROVE will carry 
out refinement until the number of iterations specified is 
reached. The first iteration is essentially a checking step and does not change the value of the parameters. 
 


\subsection{Refinement Output}


The refinement procedure produces three output files. A regular .out file with a prefix the same as the .inp file and a 
.pot file and .en file with prefixes as determined by the name given in the \verb|output| keyword in the Fitting block. 

The main output file for refinement is straightforward. The input is repeated as with other TROVE output files and then
some information is given about the eigenfunctions which were read in, etc. After this Trove prints the iteration number 
and then a list comparing the observed to calculated energies. For example
\begin{verbatim}
----------------------------------------------------------------------------
| ## |  N |  J | sym|  Obs. | Calc.| Obs.-Calc. | Weight | K     vib. quanta
---------------------------------------------------------------------------------
1  2  0  Ag  1343.5400  1346.2786  -2.7386 0.51E-03 (0) ( 0 0 0 0 0 1 0 0 0 0 0 0)*
2  3  0  Ag  1625.4000  1632.5923  -7.1923 0.26E-03 (0) ( 1 0 0 0 0 0 0 0 0 0 0 0)
3  4  0  Ag  1662.2000  1667.4972  -5.2972 0.26E-04 (0) ( 0 0 0 0 0 1 0 1 0 0 0 0)
\end{verbatim}

The first number in a row is just a label to order the output. The second is the block number which was given to a particular
state in the input file in the Fitting block. For the $A_g$ state in the example the first energy corresponding to a 
fundamental mode has a block number of 2 since 1 would correspond to the ground state with relative energy of 0. After this
the angular momentum of the state, $J$, is given along with the symmetry. The observed energy as given in the input file
is then given followed by the current iterations calculation of the energy using the adjusted potential parameters and the
difference between them. The weighting given to the state is then given. The rotational $k$ quantum number and vibrational
quantum numbers are then given. If an asterisk (*) is printed at the end of the row (as in the first row of this example)
it means that TROVE has assigned the state differently to how it was labelled in the input in the Fitting block. 

TROVE then prints a list of corrections to the potential parameters followed by the new values for the potential parameters
and the corrections rounded according to their error. 

A table is then printed which gives details on the fit for this iteration.
\begin{verbatim}
-----------------------------------------------------------------------
|  Iter | Points | Params | Deviat | ssq_ener | ssq_pot | Convergence |
-----------------------------------------------------------------------
|  1 | 18107  |   21   |  0.34175E-01 |  0.61230E+01 | 0.173E+03 | 0.293E+12 |
-------------------------------------------------------------------------------
\end{verbatim}
This gives the statistics of the fit including both the experimental energies and the \textit{ab initio} energies used
to constrain the fit. 

The Obs-Calc table and fit statistics is then repeated for each iteration.

The .en file gives similar information to the Obs-Calc table in the output file but gives calculated energies for all
states calculated by TROVE. The .pot file is a list of the \textit{ab initio} geometries with the observed (that is,
the energy given for that geometry in the file listed under geometries in the Fitting bock) energies. The calculated 
energy is also given, which is the energy given by the potential with the corrections from refinement, along with
zero-calc and the weight for the energy. 



















