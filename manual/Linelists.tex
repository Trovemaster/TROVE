
\chapter{Linelists and Spectra}
\label{chap:linelists}

This chapter will give details on how to use TROVE and associated programs to make production quality line 
lists and spectra. That is, large line lists involving millions or even billions of transitions between states. 
The programs involved are called GAIN and Exocross. Both have been designed to interface with TROVE 
outputs and using them does not require much additional syntax. 

Before discussing these programs, the selection of absorption parameters is addressed.

\section{Choosing absorption parameters}

The intensity block in TROVE requires a choice for the minimum intensities to be printed out and for the range of
states and frequencies to be included. The value of intensity thresholds should be set very small for production quality 
line lists, for example \verb|THRESH_INTES| and \verb|THRESH_LINE| can be set at $1\times 10^{-50}$ to ensure all transitions
are included.

Values for \verb|freq-window| and \verb|energy low| and \verb|upper| depend on the molecule and temperature of interest. The
lower energy range required will depend on the desired temperature range. For room temperature line lists, only relatively 
low energy states will be significantly populated. For hot line lists, this range will be increased. The partition function
for the molecule can be used to judge which states are required for coverage at a certain temperature (see below for how
to calculate using Exocross). The frequency window (and thus upper states to include) depends on the frequency of light which of interest. 

Of course, the range which is included will also be limited by practical considerations such as computational time, memory, 
basis set convergence, etc. 

\section{Intensities with GAIN}

As discussed in Chapter \ref{chap:Quickstart}, TROVE is capable of calculating transition intensities once the relevant 
eigenfunctions and dipole matrix elements have been calculated. This procedure was used in early line list papers using
TROVE.\cite{jt466,jt500,jt580}

A more efficient way of calculating intensities is to make use of the GAIN program. GAIN (\textbf{G}PU \textbf{A}ccelerated
\textbf{IN}tensities) is a program which was written by Ahmed Al-Rafaie.\cite{GAIN} 
It uses graphical processing units (GPUs) to calculate intensities far quicker than can be achieved using 
conventional TROVE. 

GAIN uses the same input file as TROVE but only the \verb|intensity| block is actually used to control the calculation. 
GAIN requires the eigenvectors, eigen description and eigen quanta files for the states of interest. It also requires the
eigen descrption and eigen quanta of the $J = 0$ state and extfield file for the dipole matrix elements (note that
currently GAIN cannot accept split dipole files, these must be stitched together). 

\subsection{Using GAIN}

The number of states for a polynomial molecule quickly increases with $J$ and energy. This leads to millions of transitions 
and so even with GAIN, intensity calculations scale quite drastically. There are a few ways in which calculations can be 
sped up however so that they can be run within wall clock limits.

The first is to increase the number of nodes used. GAIN is an mpi parallel program and can make use of multiple nodes,
which themselves have multiple cores. A rule of thumb for how many cores to use is: size of eigenvectors / memory available
per core. 

Another speed increase is to split the intensities which are being calculated by $J$ and symmetry. Rotational selection
rules limit transitions to $J'' = J'$ and $J'' = J' \pm 1$. Currently GAIN does not have a rule for only computing upper or 
lower Q branch transitions and so these duplicates should be removed for a complete line list. 
Using the selection rule, intensities can be calculated by setting
$J$ in the \verb|intensity| block to 0,1 then 1,2 then 2,3, etc. Symmetry also limits transitions but these are molecule
dependent. For example, for PF$_3$ transitions can only take place for A$_1$ $\leftrightarrow$ A$_2$ and 
E $\leftrightarrow$ E. To make use of this symmetry the nuclear statistical weights ($g_{ns}$) for the symmetries which are
allowed should be set to their usual values but others set to 0. For example for A$_1$ $\leftrightarrow$ A$_2$ in PF$_3$ the
$g_{ns}$ would be set to 8.0 8.0 0.0. For both $J$ and symmetry selection rules, a separate input file and run of GAIN
should be carried out for each selection rule.

GAIN produces two types of output files. The .out files begins with a repeat of the input file. Information is then given on
which .chk files were opened and which GPUs are being used and their memory. Information is then given on how GAIN
splits up the calculation and how many transitions are to be computed. GAIN then cycles through the energy states starting
with the lowest energy and computes all transitions to higher energies. For each complete lower energy calculation 
the current lines per second computed (L/s) is reported along with the predicted total time required. The other output
file produced is a \verb|__n__.out| file. Here \verb|n| is an integer starting at 0 going up to number of nodes $-1$. This 
file(s) contain the GAIN results and lists the frequency and the Einstein A coefficient \cite{98BuJexx} 
for a transition. Labels are also given for which states the transition is between. 

Einstein A coefficients are calculated as opposed to intensities as these are temperature (and pressure) independent. To 
simulate and plot a spectrum Exocross is used which is discussed in the next section.


Currently the format for the intensities from GAIN is not compatible with Exocross. Programs can be used however to convert
the GAIN output to the slightly more compact Exomol format.\cite{jt631} 
Code for doing this can be obtained from Sergey Yurchenko.
In the future it may be that GAIN is modified to directly output the correct format for Exocross.


\section{Exocross}

As discussed above, GAIN produces a list of temperature and pressure independent Einstein A coefficients. To simulate a 
spectra, these must be converted into intensities. This can be achieved using Exocross, providing the data is correctly
formatted. TROVE can directly produced intensities but Exocross has features which make it a better choice for production
quality simulations.

To run Exocross, two types of file are required. A TRANS file which contains information about the intensity of transitions
and a STATES file which contains the energy levels of the molecule. These files should be obtained using a program to change
GAIN output or from TROVE directly. This is likely to change depending on situation and will not be discussed here.

A simple but important calculation which can be performed using Exocross is finding the partition function at a given 
temperature. This is determined from the States file only. An example input is
\begin{verbatim}
mem 63 gb

partfunc
 ntemps 10
 tempmax 800 (K)
end

NPROCS

verbose 4

States C2H4_v01.states
\end{verbatim}
The keyword \verb|partfunc| is used to select a partition function calculation. 
\verb|ntemps| is the number of partitions of the temperature, \verb|tempmax| which will be calculated. 
For this example the partition function will be calculated at 80, 160, ... and 800 K. 
\verb|verbose| is the level of print out. \verb|States| is the name of the states file.

The output for this calculation is simple. A repeat of the input is first given and then the partition function calculation
for each temperature is given in columns. The running total of the partition function with $J$ is given in rows. 



Exocross can also be used to make a `stick spectrum'. This is an idealised spectrum where each absorption is only
represented by a line at a given wavenumber and intensity and broadening effects (doppler, collision, etc) are ignored.
An input example is
\begin{verbatim}
mem 63.0 gb

Temperature  296
Range 0 9000.0

Npoints 90001

absorption
stick

mass 28
threshold 1e-25

pf 11000.0


output C2H4_thr_1e-25_T296

ncache 1000000

NPROCS 16

verbose 4

States C2H4_v01.states

Transitions
c2h4_initial_vib_2016_intense_j0_j1__0__.out_0.-9000..trans
c2h4_initial_vib_2016_intense_j1_j2__0__.out_0.-9000..trans
.
.
\end{verbatim}
\verb|Temperature| is the temperature of interest in Kelvin. 

\verb|Range| specifies the wavelength range to be used,
in this case 0 to 9000 cm$^{-1}$. 

\verb|Npoints| controls the density of the grid produced. In this example there will be
10 points per cm$^{-1}$. 

 \verb|absoprtion| specifies that a spectra is to be computed and \verb|stick| indicates
that a stick spectrum is required. 

\verb|mass| is the molecule's mass in atomic mass units. 

\verb|threshold| is the minimum intensity of transition to be included. This is important for keeping the output file
manageable so it can be used for making plots. 

\verb|pf| is an optional keyword which is used to give the value
of the partition function rather than calculate it from the States file (the default case). This is useful if, for example,
not all $J$s have been calculated but you want to check the spectrum looks reasonable. 


\verb|output| specifies what to call the output file. 

\verb|ncache| is how much memory will be cached on the cpu during calculations. \verb|nprocs| is the number of threads
to use.

\verb|States| is the States file to use and \verb|Transitions| is a list of Trans files to use. 



Exocross has other options for simulating spectra. Examples include accounting for line broadening by using Gaussian 
or Voigt profiles for each line. The effects of particular background gas collisions can also be taken into account.
These features are fully discussed in a recent publication and manual for the Exocross program and the reader is 
directed there for full details \cite{ExoCross}.

















